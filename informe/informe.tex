\documentclass{article}
\usepackage[utf8]{inputenc}
\usepackage{graphicx}
\usepackage[simplified]{pgf-umlcd}
\usepackage{tikz}
\usepackage{multirow}
\usepackage{float}
\usetikzlibrary{positioning,fit,calc,arrows.meta, shapes}
\usepackage{wrapfig}
\usepackage{amsmath}
\usepackage{mathtools}
\graphicspath{ {images/} }
\usepackage{array}
\newcolumntype{L}[1]{>{\raggedright\let\newline\\\arraybackslash\hspace{0pt}}m{#1}}
\newcolumntype{C}[1]{>{\centering\let\newline\\\arraybackslash\hspace{0pt}}m{#1}}
\newcolumntype{R}[1]{>{\raggedleft\let\newline\\\arraybackslash\hspace{0pt}}m{#1}}

%Tot això hauria d'anar en un pkg, però no sé com és fa
\newcommand*{\assignatura}[1]{\gdef\1assignatura{#1}}
\newcommand*{\grup}[1]{\gdef\3grup{#1}}
\newcommand*{\professorat}[1]{\gdef\4professorat{#1}}
\renewcommand{\tablename}{Taula}
\renewcommand{\title}[1]{\gdef\5title{#1}}
\renewcommand{\author}[1]{\gdef\6author{#1}}
\renewcommand{\date}[1]{\gdef\7date{#1}}
\renewcommand{\contentsname}{Índex}
\renewcommand{\listtablename}{Llista de taules}
\renewcommand{\maketitle}{ %fa el maketitle de nou
    \begin{titlepage}
        \raggedright{UNIVERSITAT DE LLEIDA \\
            Escola Politècnica Superior \\
            Grau en Enginyeria Informàtica\\
            \1assignatura\\}
            \vspace{5cm}
            \centering\huge{\5title \\}
            \vspace{3cm}
            \large{\6author} \\
            \normalsize{\3grup}
            \vfill
            Professorat : \4professorat \\
            Data : \7date
\end{titlepage}}
%Emplenar a partir d'aquí per a fer el títol : no se com es fa el package
%S'han de renombrar totes, inclús date, si un camp es deixa en blanc no apareix

\tikzset{
	%Style of nodes. Si poses aquí un estil es pot reutilitzar més facilment
	base/.style = {circle, draw=black,
      minimum width=0.75cm, font=\ttfamily,
      text centered},
    dots/.style = {circle, draw=white,
      minimum width=0.75cm, font=\ttfamily,
      text centered},
    last/.style = {base, fill=orange!15},
    remove/.style = {base, fill=red!15},
    change/.style = {base, fill=green!15},
    tree/.style = {base, rectangle, minimum height=0.75cm},
    stack/.style = {rectangle, font=\ttfamily, rounded corners, draw=black,
      minimum width=4cm, minimum height=1cm,
      text centered},
   	even/.style = {stack, fill=green!30},
   	odd/.style = {stack, fill=orange!15},
   	blank/.style = {stack, minimum height=0.5cm, draw=white},
   	typetag/.style={rectangle, draw=black!50, font=\ttfamily, anchor=west}
}
\renewcommand{\figurename}{Figura}
\title{Pràctica: Jerarquia de Memòria}
\author{Sergi Sales Jové, Sergi Simón Balcells}
\date{Dimecres 24 de Abril}
\assignatura{Arquitectura de Computadors}
\professorat{Concepció Roig}
\grup{GM3}

%Comença el document

\begin{document}
\maketitle
\thispagestyle{empty}

\newpage
\pagenumbering{roman}
\tableofcontents
\listoftables
\newpage
\pagenumbering{arabic}
\section{Introducció}
Per veure el funcionament del processament segmentat es va fer un programa que realitzes una petita fórmula matemàtica composta pel sumatori d'un seguit de divisions d'exponents. El programa es va realitzar fora de classe de dues formes, una forma bàsica i una segona forma apta per aplicar la millora Delay Slot en executar-lo.\\
Amb els programes creats i verificats, en el laboratori es va realitzar l'execució d'aquests amb el software winmips64, que ens deixava executar-lo amb el processament segmentat amb l'opció d'afegir un parell d'opcions de millora:
\begin{itemize}
\item[--]{Delay slot:} Permet que algunes instruccions s'executin sense els efectes de les instruccions anteriors. Això passa, per exemple, en una instrucció senzilla col·locada després d'una instrucció de salt.
\item[--]{Forwarding:} És una opció que optimitza l'execució del programa reduint els riscos que pot provocar el pipelining.
\end{itemize}
Els resultats de les execucions es troben en la Taula \ref{tab:results} de l'Annex.
\section{Anàlisi de resultats}
\subsection{Opcions de millora}
L'execució del programa es podia realitzar variant les opcions de millora que ens proposava el winmips64. Aquestes eren Delay slot i Forwarding. Aleshores com es demanava a la pràctica, vàrem executar el programa sense opcions de millora, amb només una de les dues (en els dos casos) i amb les dues alhora, com es pot veure a la Taula \ref{tab:results}.\\
Si ens fixem en la primera columna, on tenim l'execució sense opcions de millora, podem veure que el CPI és més gran que 5. En principi les instruccions tenen 5 cicles per cada instrucció, i això vol dir que en executar-ho de manera segmentada acabem tenint un CPI més gran que l'execució seqüencial? Doncs no. Resulta que durant l'execució, algunes de les operacions acaben executant-se amb molts més cicles, com la divisió.\\
Si comparem les altres columnes de la taula, podem veure que les columnes amb menys cicles són les quals tenen activat el Forwarding. Aquestes dues columnes tenen el mateix CPI, i en l'únic que es diferencien és en els RAW stalls, i això deu ser conseqüència del Delay slot activat en la 4a columna.\\
Aquest fenomen pot ser perquè el programa estava ben optimitzat i no presentava quasi diferencia amb la versió desenvolupada per al delay.\\
Una altra cosa a observar és el nombre de cicles de cada execució, que ens mostren quina d'aquestes finalitza abans que les altres. En aquest cas, amb l'opció de Forwarding els programes acaben més de pressa (acaben igual en els dos casos de Forwarding), tot i que el cas del delay slot no s'allunya tant com el cas que no té millores.\\
Un dels últims criteris que tindrem en compte és l'acceleració d'aquestes execucions, que es calcula amb la següent fórmula:\\
$$ acceleracio = \frac{CPI\ ideal * profunditat\ de\ segmentacio}{CPI\ ideal + cicles\ de\ detencio} $$\\
Els càlculs del Speedup respecte a l'execució seqüencial per cada cas estudiat són les següents:
\begin{itemize}
\item[--]{Sense opcions de millora}
$$ acceleracio = \frac{1 * 5}{1 + \frac{144+15\ detencions}{196\ cicles}}=2.76 $$
\item[--]{Forwarding}
$$ acceleracio = \frac{1 * 5}{1 + \frac{145\ detencions}{139\ cicles}}=2.44 $$
\item[--]{Delay slot}
$$ acceleracio = \frac{1 * 5}{1 + \frac{111\ detencions}{152\ cicles}}=2.89 $$
\item[--]{Delay slot + forwarding}
$$ acceleracio = \frac{1 * 5}{1 + \frac{120\ detencions}{139\ cicles}}=2.68 $$
\end{itemize}
Com es pot veure aplicant el delay slot aconseguim l'acceleració més gran, ja que és el que té l'índex de detencions/cicles més petit.

\section{Conclusió}
Veient els resultats obtinguts durant el laboratori, i els resultats obtinguts al fer el càlcul de l'acceleració podem decidir quina execució ens interessa més.\\
Creiem que la millor execució és la del Delay slot, sense Forwarding, ja que és la que aconsegueix una millor acceleració, diferenciant-se bastant de les execucions amb forwarding. A més a més, la diferència entre el CPI del delay slot amb les dues execucions amb Forwarding no es diferencien massa. Un altre dels criteris a tenir en compte en la decisió és en nombre de "stalls" que ocasionen aquestes execucions, i en aquest cas, el Delay és la que menys en genera.\\
Per tot el que hem comentat en el paràgraf anterior, creiem que el Delay slot, sense el Forwarding, és la millor opció, ja que és ràpid, i no genera tants conflictes com la resta.

\newpage
\section{Annex}

%TODO vull centrar la taula pero no puc.

\begin{table}[!h]
\hskip-2.0cm
\begin{tabular}{|c|c|c|c|c|}

\hline
 & Sense opcions de millora &Forwarding  &Delay Slot  &Forwarding + Delay Slot   \\
 \hline
 Nombre d'instruccions &37  &37  &37 &37  \\
 \hline
 Nombre de cicles &196               &139  &152  &139   \\
 \hline
 CPI &5.297               &3.757  &4.108  &3.757   \\
 \hline
 RAW stalls &144 &140  &111  &115   \\
 \hline
 WAW stalls &0 &0 &0 &0   \\
 \hline
 WAR stalls &0 &4 &0 &4   \\
 \hline
 Structural stalls &15 &1 &0 &1\\
 \hline 
\end{tabular}
\caption{Resultats de l'experiment}
\label{tab:results}
\end{table}
\end{document}